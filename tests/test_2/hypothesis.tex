\documentclass[11pt]{article}
\usepackage[margin=1in]{geometry}
\usepackage{amsmath,amssymb,amsthm}
\usepackage{hyperref}
\usepackage{enumitem}
\usepackage{cleveref}

\newtheorem{theorem}{Theorem}[section]
\newtheorem{lemma}[theorem]{Lemma}
\newtheorem{corollary}[theorem]{Corollary}
\newtheorem{definition}[theorem]{Definition}

\title{\textbf{Hypothesis for Test 2}}
\author{AI Test Framework}
\date{\today}

\begin{document}
\maketitle

\section*{\textbf{Hypothesis}}

Based on the foundational truths and the rules outlined in `theline.tex`, the hypothesis for Test 2 is as follows:

\begin{itemize}
    \item The Collatz sequence for an odd starting number greater than 1 will eventually reach 1.
    \item The sequence will not include any even numbers as starting values.
    \item The sequence will follow the pattern of reducing even numbers to odd numbers through division by 2.
    \item The sequence will exhibit structural relationships and patterns as described in the foundational truths.
\end{itemize}

\section*{\textbf{Test Objective}}

The objective of this test is to validate the hypothesis by running the Collatz sequence for an odd starting number and observing the sequence's behavior. The test will ensure that the sequence adheres to the rules and patterns outlined in the hypothesis.

\section*{\textbf{Expected Outcome}}

The expected outcome of this test is that the Collatz sequence for the given odd starting number will eventually reach 1, following the rules and patterns described in the hypothesis.

\end{document}
